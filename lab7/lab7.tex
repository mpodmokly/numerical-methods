\documentclass[11pt]{scrartcl}
\usepackage{polski}
\usepackage[polish]{babel}

\usepackage{graphicx, float, caption, subcaption, amsmath}
\usepackage{tabularx, multirow, hyperref, enumitem, listings}
\usepackage{xcolor}
%\usepackage{minted}

\hypersetup{
    colorlinks=true,
    linkcolor=black,
    urlcolor=black,
    citecolor=black
}

\definecolor{md-black}{rgb}{0.12, 0.12, 0.12}
\definecolor{md-teal}{rgb}{0.38, 0.79, 0.69}
\definecolor{md-mauve}{rgb}{0.76, 0.52, 0.75}
\definecolor{md-yellow}{rgb}{0.86, 0.86, 0.67}
\definecolor{md-green}{rgb}{0.13, 0.55, 0.13}
\definecolor{md-red}{rgb}{0.82, 0.10, 0.14}
\definecolor{md-purple}{rgb}{0.69, 0.33, 0.73}
\definecolor{md-orange}{rgb}{0.96, 0.42, 0.18}
\definecolor{md-gray}{rgb}{0.44, 0.46, 0.51}
\lstset{
    language=Python,
    basicstyle=\color{md-teal}\ttfamily,
    keywordstyle=\color{md-mauve},
    commentstyle=\color{md-green},
    stringstyle=\color{md-red},
    numbers=left,
    numberstyle=\small\color{md-gray}\ttfamily,
    stepnumber=1,
    numbersep=5pt,
    backgroundcolor=\color{md-black},
    showspaces=false,
    showstringspaces=false,
    showtabs=false,
    frame=none,
    tabsize=4,
    captionpos=b,
    breaklines=true,
    breakatwhitespace=false,
    escapeinside={\%*}{*)},
    numbersep=-10pt,
    morekeywords={as},
    classoffset=1,
    morekeywords={quad, quad_vec, trapz, simps, linregress},
    keywordstyle=\color{md-yellow},
    classoffset=0
}

\graphicspath{{../images/}}

\title{Laboratorium 7 - Kwadratury adaptacyjne}
\author{Mateusz Podmokły - II rok Informatyka WI}
\date{18 kwiecień 2024}

\begin{document}
    \maketitle
    \section{Treść zadania}
    \textbf{Zadanie 1.} Oblicz wartość całki
    \[
        \int_{0}^{1}\frac{4}{1+x^2}dx=\pi
    \]
    korzystając z:
    \begin{itemize}
        \item kwadratur adaptacyjnych trapezów,
        \item kwadratur adaptacyjnych Gaussa-Kronroda.
    \end{itemize}
    Dla każdej metody narysuj wykres wartości bezwzględnej
    błędu względnego w zależności od liczby ewaluacji funkcji
    podcałkowej. Przyjmij wartości tolerancji z zakresu od
    $10^0$ do $10^{-14}$.

    \subsection*{}
    \textbf{Zadanie 2.} Powtórz obliczenia z poprzedniego oraz
    dzisiejszego laboratorium dla całek
    \[
        \int_{0}^{1}\sqrt{x}logxdx=-\frac{4}{9}
    \]
    oraz
    \[
        \int_{0}^{1}\left( \frac{1}{(x-0.3)^2+a} + \frac{1}
            {(x-0.9)^2+b} - 6 \right)dx
    \]
    Przyjmij $a=0.001$ oraz $b=0.004$. Wykorzystaj fakt, że
    \[
        \int_{0}^{1}\frac{1}{(x-x_0)^2+a}dx=\frac{1}{\sqrt{a}}
            \left( arctg \frac{1-x_0}{\sqrt{a}} + arctg
            \frac{x_0}{\sqrt{a}} \right)
    \]

    \section{Specyfikacja użytego środowiska}
    Specyfikacja:

    \begin{itemize}
        \item Środowisko: Visual Studio Code,
        \item Język programowania: Python,
        \item System operacyjny: Microsoft Windows 11,
        \item Architektura systemu: x64.
    \end{itemize}

    \section{Rozwiązanie problemu}
    \subsection{Biblioteki}
    W realizacji rozwiązania wykorzystane zostały następujące
    biblioteki:
    \begin{lstlisting}
        import numpy as np
        from matplotlib import pyplot as plt
        from scipy.integrate import quad_vec
    \end{lstlisting}

    \subsection{Zadanie 1.}
    Do obliczenia całki metodą adaptacyjną trapezów
    i Gaussa-Kronroda wykorzystana została funkcja
    z biblioteki \texttt{SciPy}
    \[
        \texttt{scipy.integrate.quad\_vec}
    \]
    z parametrem $\texttt{epsrel} \in [10^0,10^{-14}]$.

    \subsection{Zadanie 2.}

\end{document}
