\documentclass[11pt, leqno]{scrartcl}
\usepackage{polski}
\usepackage[polish]{babel}

\usepackage{graphicx, float, caption, subcaption}
\usepackage{tabularx, multirow, hyperref, enumitem}
\usepackage{listings, xcolor}
\usepackage{amsmath, amssymb}
%\usepackage{minted}

\hypersetup{
    colorlinks=true,
    linkcolor=black,
    urlcolor=black,
    citecolor=black
}

\definecolor{md-black}{rgb}{0.12, 0.12, 0.12}
\definecolor{md-teal}{rgb}{0.38, 0.79, 0.69}
\definecolor{md-mauve}{rgb}{0.76, 0.52, 0.75}
\definecolor{md-yellow}{rgb}{0.86, 0.86, 0.67}
\definecolor{md-green}{rgb}{0.13, 0.55, 0.13}
\definecolor{md-red}{rgb}{0.82, 0.10, 0.14}
\definecolor{md-purple}{rgb}{0.69, 0.33, 0.73}
\definecolor{md-orange}{rgb}{0.96, 0.42, 0.18}
\definecolor{md-gray}{rgb}{0.44, 0.46, 0.51}
\lstset{
    language=Python,
    basicstyle=\color{md-teal}\ttfamily,
    keywordstyle=\color{md-mauve},
    commentstyle=\color{md-green},
    stringstyle=\color{md-red},
    numbers=left,
    numberstyle=\small\color{md-gray}\ttfamily,
    stepnumber=1,
    numbersep=5pt,
    backgroundcolor=\color{md-black},
    showspaces=false,
    showstringspaces=false,
    showtabs=false,
    frame=none,
    tabsize=4,
    captionpos=b,
    breaklines=true,
    breakatwhitespace=false,
    escapeinside={\%*}{*)},
    numbersep=-10pt,
    morekeywords={as},
    classoffset=1,
    morekeywords={quad, quad_vec, trapz, simps, linregress,
        newton},
    keywordstyle=\color{md-yellow},
    classoffset=0
}

\graphicspath{{../images/}}

\title{Laboratorium 11 - Optymalizacja}
\author{Mateusz Podmokły - II rok Informatyka WI}
\date{6 czerwiec 2024}

\begin{document}
    \maketitle
    \section{Treść zadania}
    \textbf{Zadanie 1.} Wyznacz punkty krytyczne każdej z poniższych
    funkcji. Scharakteryzuj każdy znaleziony punkt jako minimum,
    maksimum lub punkt siodłowy. Dla każdej funkcji zbadaj, czy
    posiada minimum globalne lub maksimum globalne na zbiorze
    $\mathbb{R}^2$.
    \begin{align*}
        &f_1(x,y)=x^2-4xy+y^2 \\
        &f_2(x,y)=x^4-4xy+y^4 \\
        &f_3(x,y)=2x^3-3x^2-6xy(x-y-1) \\
        &f_4(x,y)=(x-y)^4+x^2-y^2-2x+2y+1
    \end{align*}

    \subsection*{}
    \textbf{Zadanie 2.}

    \section{Specyfikacja użytego środowiska}
    Specyfikacja:
    \begin{itemize}
        \item Środowisko: Visual Studio Code,
        \item Język programowania: Python,
        \item System operacyjny: Microsoft Windows 11,
        \item Architektura systemu: x64.
    \end{itemize}

    \section{Rozwiązanie problemu}
    \subsection{Biblioteki}
    W realizacji rozwiązania wykorzystane zostały następujące
    biblioteki:
    \begin{lstlisting}
    import numpy as np
    import matplotlib.pyplot as plt
    \end{lstlisting}

    \subsection{}

\end{document}
