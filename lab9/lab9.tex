\documentclass[11pt, leqno]{scrartcl}
\usepackage{polski}
\usepackage[polish]{babel}

\usepackage{graphicx, float, caption, subcaption}
\usepackage{tabularx, multirow, hyperref, enumitem}
\usepackage{listings, xcolor}
\usepackage{amsmath, amssymb}
%\usepackage{minted}

\hypersetup{
    colorlinks=true,
    linkcolor=black,
    urlcolor=black,
    citecolor=black
}

\definecolor{md-black}{rgb}{0.12, 0.12, 0.12}
\definecolor{md-teal}{rgb}{0.38, 0.79, 0.69}
\definecolor{md-mauve}{rgb}{0.76, 0.52, 0.75}
\definecolor{md-yellow}{rgb}{0.86, 0.86, 0.67}
\definecolor{md-green}{rgb}{0.13, 0.55, 0.13}
\definecolor{md-red}{rgb}{0.82, 0.10, 0.14}
\definecolor{md-purple}{rgb}{0.69, 0.33, 0.73}
\definecolor{md-orange}{rgb}{0.96, 0.42, 0.18}
\definecolor{md-gray}{rgb}{0.44, 0.46, 0.51}
\lstset{
    language=Python,
    basicstyle=\color{md-teal}\ttfamily,
    keywordstyle=\color{md-mauve},
    commentstyle=\color{md-green},
    stringstyle=\color{md-red},
    numbers=left,
    numberstyle=\small\color{md-gray}\ttfamily,
    stepnumber=1,
    numbersep=5pt,
    backgroundcolor=\color{md-black},
    showspaces=false,
    showstringspaces=false,
    showtabs=false,
    frame=none,
    tabsize=4,
    captionpos=b,
    breaklines=true,
    breakatwhitespace=false,
    escapeinside={\%*}{*)},
    numbersep=-10pt,
    morekeywords={as},
    classoffset=1,
    morekeywords={quad, quad_vec, trapz, simps, linregress,
        newton},
    keywordstyle=\color{md-yellow},
    classoffset=0
}

\graphicspath{{../images/}}

\title{Laboratorium 9 - Równania różniczkowe zwyczajne}
\author{Mateusz Podmokły - II rok Informatyka WI}
\date{16 maj 2024}

\begin{document}
    \maketitle
    \section{Treść zadania}
    \textbf{Zadanie 1.} Przedstaw każde z poniższych równań
    różniczkowych zwyczajnych jako równoważny układ równań
    pierwszego rzędu (ang. first-order system of ODEs):
    \begin{enumerate}
        \item równanie Van der Pol'a:
            \[
                y''=y'(1-y^2)-y
            \]
        \item równanie Blasiusa:
            \[
                y'''=-yy''
            \]
        \item II zasada dynamiki Newtona dla problemu dwóch
            ciał:
            \[
                y_1''=-\frac{GMy_1}{(y_1^2+y_2^2)^{\frac{3}{2}}}
            \]
            \[
                y_2''=-\frac{GMy_2}{(y_1^2+y_2^2)^{\frac{3}{2}}}
            \]
    \end{enumerate}

    \subsection*{}
    \textbf{Zadanie 2.} Dane jest równanie różniczkowe
    zwyczajne
    \[
        y'=-5y
    \]
    z warunkiem początkowym $y(0) = 1$. Równanie rozwiązujemy
    numerycznie z krokiem $h = 0.5$.
    \begin{enumerate}
        \item Analityczna stabilność. Wyjaśnij, czy rozwiązania
            powyższego równania są stabilne?
        \item Numeryczna stabilność. Wyjaśnij, czy metoda
            Euler'a jest stabilna dla tego równania z użytym
            krokiem $h$?
        \item Oblicz numerycznie wartości przybliżonego
            rozwiązania dla $t = 0.5$ metodą Euler'a.
        \item Wyjaśnij, czy niejawna metoda Euler'a jest
            stabilna dla tego równania z użytym krokiem $h$?
        \item Oblicz numerycznie wartości przybliżonego
            rozwiązania dla $t = 0.5$ niejawną metodą Euler'a.
    \end{enumerate}

    \subsection*{}
    \textbf{Zadanie 3.}
    

    \section{Specyfikacja użytego środowiska}
    Specyfikacja:
    \begin{itemize}
        \item Środowisko: Visual Studio Code,
        \item Język programowania: Python,
        \item System operacyjny: Microsoft Windows 11,
        \item Architektura systemu: x64.
    \end{itemize}

    \section{Rozwiązanie problemu}
    \subsection{Biblioteki}
    W realizacji rozwiązania wykorzystane zostały następujące
    biblioteki:
    \begin{lstlisting}
        import numpy as np
    \end{lstlisting}

    \subsection{Zadanie 1.}
    Każde równanie w postaci równania różniczkowego zwyczajnego
    zostanie przekształcone równoważnie do układu równań
    pierwszego rzędu.

    \subsubsection{Równanie 1.}
    Mamy równanie
    \[
        y''=y'(1-y^2)-y
    \]
    Zastosujemy podstawienie
    \[
        z=y'
    \]
    Otrzymuejmy równoważny układ równań
    \[
        \begin{cases}
            y'=z \\
            z'=z(1-y^2)-y
        \end{cases}
    \]

    \subsubsection{Równanie 2.}
    Mamy równanie
    \[
        y'''=-yy''
    \]
    Stosujemy analogiczne podstawienia
    \[
        z=y'
    \]
    \[
        w=z'
    \]
    Otrzymuejmy układ równań
    \[
        \begin{cases}
            y'=z \\
            z'=w \\
            w'=-yw
        \end{cases}
    \]

    \subsubsection{Równanie 3.}
    Mamy równanie
    \[
        y_1''=-\frac{GMy_1}{(y_1^2+y_2^2)^{\frac{3}{2}}}
    \]
    \[
        y_2''=-\frac{GMy_2}{(y_1^2+y_2^2)^{\frac{3}{2}}}
    \]
    Stosujemy analogiczne podstawienia
    \[
        z_1=y_1'
    \]
    \[
        z_2=y_2'
    \]
    Otrzymuejmy układ równań
    \[
        \begin{cases}
            y_1'=z_1 \\
            y_2'=z_2 \\
            z_1'=-\frac{GMy_1}{(y_1^2+y_2^2)^{\frac{3}{2}}} \\
            z_2'=-\frac{GMy_2}{(y_1^2+y_2^2)^{\frac{3}{2}}}
        \end{cases}
    \]

    \subsection{Zadanie 2.}
    \subsubsection{Analityczna stabilność}
    Mamy równanie
    \[
        y'=-5y
    \]
    z warunkiem początkowym
    \[
        y(0)=1
    \]
    Możemy je rozwiązać analitycznie w prosty sposób metodą
    separacji zmiennych. Otrzymujemy rozwiązanie
    \[
        y(t)=y_0e^{-5t}
    \]
    gdzie
    \[
        y_0=y(0)=1
    \]
    Teraz sprawdźmy stabilność otrzymanego rozwiązania.
    Równanie różniczkowe jest stabilne, jeśli dla każdego
    $\epsilon > 0$ istnieje $\delta > 0$ takie, że jeśli
    początkowe odchylenie
    \[
        |y(0)-y_0|<\delta
    \]
    to
    \[
        |y(t)-\hat{y}(t)|<\epsilon
    \]
    dla każdego $t \geq 0$. \\ 
    Rozważmy zaburzenie początkowego warunku
    \[
        y(0)=y_0
    \]
    Niech
    \[
        \hat{y}(0)=y_0+\delta
    \]
    Mamy wtedy rozwiązanie równania dla początkowego warunku
    \[
        \hat{y}(t)=(y_0+\delta)e^{-5t}
    \]
    Możemy podstawić
    \[
        |y(t)-\hat{y}(t)|=|y_0e^{-5t}-(y_0+\delta)e^{-5t}|=
            |\delta e^{-5t}|
    \]
    Wiadomo, że $e^{-5t} \leq 1$ dla każdego $t \geq 0$, więc
    \[
        |\delta e^{-5t}| \leq |\delta|
    \]
    Możemy wtedy wybrać $\delta$ takie, że
    $|\delta | \leq \epsilon $. Wtedy
    \[
        |y(t)-\hat{y}(t)| < \epsilon
    \]
    Oznacza to, że rozwiązania tego równania są stabilne.

    \subsubsection{Numeryczna stabilność}
    Dla równań różniczkowych postaci
    \[
        y'=\lambda y
    \]
    gdzie $\lambda \in \mathbb{R}$, warunek stabilności
    numerycznej metody Euler'a dany jest wzorem
    \[
        |1+h\lambda|<1
    \]
    Mamy $\lambda =-5$, więc po przekształceniu równoważnym
    podanego warunku otrzymujemy
    \[
        0<h<0.4
    \]
    Z tego wynika, że dla tego równania z krokiem $h=0.5$
    metoda Euler'a nie jest stabilna numerycznie.

    \subsubsection{Metoda Euler'a}
    Rozwiązanie równania postaci
    \[
        y'=f(y,t)
    \]
    o warunku początkowym $y(t_0)=y_0$ i kroku $h$, z punktami
    $t_i$ na osi OX:
    \[
        t_{n+1}=t_n+h,\quad n=0,1,2,\dots
    \]
    za pomocą metody Euler'a możemy otrzymać ze wzoru:
    \[
        y_{n+1}=y_n+hf(y_n,t_n)
    \]
    
    \subsubsection{Stabilność niejawnej metody Euler'a}
    Dla równań różniczkowych postaci
    \[
        y'=\lambda y
    \]
    gdzie $\lambda \in \mathbb{R}$, warunek stabilności
    numerycznej niejawnej metody Euler'a dany jest wzorem
    \[
        \left| \frac{1}{1-h\lambda} \right| <1
    \]
    Mamy $\lambda =-5$, więc po przekształceniu równoważnym
    podanego warunku otrzymujemy
    \[
        h>0
    \]
    Z tego wynika, że dla dowolnego kroku $h>0$ niejawna metoda
    Euler'a jest stabilna numerycznie.

    \subsubsection{Niejawna metoda Euler'a}
    Rozwiązanie równania postaci
    \[
        y'=f(y,t)
    \]
    o warunku początkowym $y(t_0)=y_0$ i kroku $h$, z punktami
    $t_i$ na osi OX:
    \[
        t_{n+1}=t_n+h,\quad n=0,1,2,\dots
    \]
    za pomocą niejawnej metody Euler'a możemy otrzymać ze
    wzoru:
    \[
        y_{n+1}=y_n+hf(y_{n+1},t_{n+1})
    \]
    Mamy równanie
    \[
        y'=-5y
    \]
    o warunku początkowym $y(0)=1$ i kroku $h=0.5$.
    Podstawiamy $f(y,t)$ do wzoru
    \[
        y_{n+1}=y_n+h(-5y_{n+1})
    \]
    Przekształcając równoważnie otrzymujemy wzór na $y_{n+1}$
    \[
        y_{n+1}=\frac{y_n}{5h+1}
    \]
    
    \section{Przedstawienie wyników}
    \subsection{Jawna metoda Euler'a}
    Dla równania
    \[
        y'=-5y
    \]
    oraz $y(0)=1$ i $h=0.5$ otrzymujemy wynik
    \[
        y(0.5)=-1.5
    \]
    a więc różni się on znacznie od tego wyznaczonego
    analitycznie, który wynosi
    \[
        y_{true}(0.5) \approx 0.082
    \]
    Rozbieżność wynika prawdopodobnie z braku stabilności
    numerycznej tej metody z przyjętym krokiem $h$. Dla
    $h=0.1$ wynik to już
    \[
        y(0.5) \approx 0.031
    \]

    \subsection{Niejawna metoda Euler'a}
    Po wykonaniu obliczeń z wykorzystaniem wyprowadzonego
    wzoru otrzymujemy wynik
    \[
        y(0.5) \approx 0.286
    \]
    Jest on znacznie lepszy niż w przypadku jawnej
    metody Euler'a dla kroku $h=0.5$. Zmieńmy krok
    na $h=0.1$. Otrzymujemy wtedy wynik
    \[
        h(0.5) \approx 0.132
    \]

    \section{Wnioski}
    Niejawna metoda Euler'a bardziej stabilna, ale mniej
    dokładna.

    \section{Bibliografia}
    \url{https://pl.wikipedia.org/wiki/Metoda_Eulera}
    
\end{document}
