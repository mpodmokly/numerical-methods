\documentclass[11pt]{scrartcl}
\usepackage{polski}
\usepackage[polish]{babel}

\usepackage{graphicx, float, caption, subcaption}
\usepackage{amsmath}
\usepackage{tabularx}
\usepackage{multirow}
\usepackage{hyperref}
\usepackage{enumitem}
%\usepackage{minted}
\graphicspath{{images/}}

\title{Laboratorium 2 - Regresja liniowa metodą najmniejszych kwadratów}
\author{Mateusz Podmokły}
\date{7 marzec 2024}

\begin{document}
    \maketitle

    \section{Treść zadania}
    \textbf{Zadanie 1.} Celem zadania jest zastosowanie metody najmniejszych kwadratów
    do predykcji, czy nowotwór jest złosliwy, czy łagodny. Nowotwory złosliwe i łagodne
    maja rózne charakterystyki wzrostu. \\
    Do rozwiazania problemu wykorzystamy bibliotekę \texttt{pandas}, typ
    \texttt{DataFrame} oraz dwa zbiory danych:
    \begin{itemize}[label=--]
        \item \texttt{breast-cancer-train.dat}
        \item \texttt{breast-cancer-validate.dat}
    \end{itemize}
    Zawierają one klasę nowotworu oraz cechy, tj. charakterystyki nowotworu. \\
    Wykorzystamy liniową oraz kwadratową metodę najmniejszych kwadratów. Do
    reprezentacji danych liniowej metody najmniejszych kwadratów wykorzystamy macierz
    \[
        A_{lin}=
    \]

    \section{Specyfikacja użytego środowiska}
    Specyfikacja:

    \begin{itemize}
        \item Środowisko: Visual Studio Code,
        \item Język programowania: Python,
        \item System operacyjny: Microsoft Windows 11,
        \item Architektura systemu: x64.
    \end{itemize}

    \section{Rozwiązanie problemu}

\end{document}
