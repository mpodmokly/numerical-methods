\documentclass[11pt]{scrartcl}
\usepackage{polski}
\usepackage[polish]{babel}

\usepackage{graphicx, float, caption, subcaption, amsmath}
\usepackage{tabularx, multirow, hyperref, enumitem, listings}
%\usepackage{minted}

\usepackage{listings, xcolor}
\definecolor{md-black}{rgb}{0.12, 0.12, 0.12}
\definecolor{md-teal}{rgb}{0.38, 0.79, 0.69}
\definecolor{md-mauve}{rgb}{0.76, 0.52, 0.75}
\definecolor{md-green}{rgb}{0.13, 0.55, 0.13}
\definecolor{md-red}{rgb}{0.82, 0.10, 0.14}
\definecolor{md-purple}{rgb}{0.69, 0.33, 0.73}
\definecolor{md-orange}{rgb}{0.96, 0.42, 0.18}
\definecolor{md-gray}{rgb}{0.44, 0.46, 0.51}

\lstset{
    language=Python,
    basicstyle=\color{md-teal}\ttfamily,
    keywordstyle=\color{md-mauve},
    commentstyle=\color{md-green},
    stringstyle=\color{md-red},
    numbers=left,
    numberstyle=\small\color{md-gray}\ttfamily,
    stepnumber=1,
    numbersep=5pt,
    backgroundcolor=\color{md-black},
    showspaces=false,
    showstringspaces=false,
    showtabs=false,
    frame=none,
    tabsize=4,
    captionpos=b,
    breaklines=true,
    breakatwhitespace=false,
    escapeinside={\%*}{*)},
    numbersep=-10pt,
    morekeywords={as}
}

\graphicspath{{images/}}

\title{Laboratorium 4 - Efekt Rungego}
\author{Mateusz Podmokły - II rok Informatyka WI}
\date{21 marzec 2024}

\begin{document}
    \maketitle

    \section{Treść zadania}
    \textbf{Zadanie 1.} Wyznacz wielomiany interpolujące funkcje:
    \[
        f_1(x)=\frac{1}{1+25x^2}, x \in [-1,1],
    \]
    \[
        f_2(x)=e^{cos(x)}, x \in [0,2\pi],
    \]
    używając:
    \begin{itemize}[label=--]
        \item wielomianów Lagrange'a z węzłami $x_j=x_0+jh, j=0,1,
            \ldots ,n, h=\frac{x_n-x_0}{n}$
        \item kubicznych funkcji sklejancyh z węzłami $x_j=x_0+jh,
            j=0,1, \ldots ,n, h=\frac{x_n-x_0}{n}$
        \item wielomianów Lagrange'a z węzłami Czebyszewa
        \[
            x_j=cos(\theta_j)
        \]
        \[
            \theta_j=\frac{2j+1}{2(n+1)}\pi, 0 \leq j \leq n.
        \]
    \end{itemize}
    Dla funkcji Rungego $f_1(x)$ wykonaj interpolację podanymi
    sposobami z $n=12$ węzłami interpolacji. Przedstaw na wykresie
    funkcję $f_1(x)$ oraz wyniki interpolacji. \\
    Wykonaj interpolację funkcji $f_1(x)$ i $f_2(x)$ podanymi
    sposobami z $n=4,5, \ldots ,50$ węzłami interpolacji. Przeprowadź
    ewaluację wyników na zbiorze 500 losowo wybranych punktów.
    Na wykresie przedstaw normę wektora błędu na tym zbiorze
    punktów w zależności od liczby węzłów interpolacji dla każdej
    metody, osobno dla obudwu funkcji.

    \section{Specyfikacja użytego środowiska}
    Specyfikacja:

    \begin{itemize}
        \item Środowisko: Visual Studio Code,
        \item Język programowania: Python,
        \item System operacyjny: Microsoft Windows 11,
        \item Architektura systemu: x64.
    \end{itemize}

    \section{Rozwiązanie problemu}
    W realizacji rozwiązania wykorzystane zostały następujące
    biblioteki:
    \begin{lstlisting}
        import numpy as np
        import matplotlib.pyplot as plt
    \end{lstlisting}

    \subsection*{}
    Obliczamy wielomian interpolacyjny Lagrange'a ze wzoru
    \[
        w(x)=\sum_{i=0}^{n}y_i \cdot \prod_{j=0 \land j \neq i}^{n}
        \frac{x-x_j}{x_i-x_j}
    \]
    Funkcje sklejane (splajny) obliczamy na każdym przedziale
    oddzielnie wykorzystując wielomiany Lagrange'a. Węzły
    Czebyszewa na przedziale $[-1,1]$ obliczamy ze wzoru
    \[
        x_j=cos(\theta_j)
    \]
    \[
        \theta_j=\frac{2j+1}{2(n+1)}\pi, 0 \leq j \leq n.
    \]
    Transformacja węzłów Czebyszewa $r \in [-1,1]$ na punkty
    $x \in [a,b]$ dana jest wzorem
    \[
        x=a+ \frac{(b-a)(r+1)}{2}
    \]
    Punkty do ewaluacji wylosowane zostały z użyciem funkcji
    \texttt{np.random.uniform}, następnie obliczona została norma
    wektora błędu względnego interpolacji za pomocą funkcji
    \texttt{np.linalg.norm}.
    
    \section{Przedstawienie wyników}


\end{document}
