\documentclass[11pt]{scrartcl}
\usepackage{polski}
\usepackage[polish]{babel}

\usepackage{graphicx, float, caption, subcaption}
\usepackage{amsmath}
\usepackage{tabularx}
\usepackage{multirow}
\usepackage{hyperref}
\usepackage{enumitem}
\usepackage{amsmath}
\usepackage{listings}
%\usepackage{minted}
\graphicspath{{images/}}

\title{Laboratorium 3 - Interpolacja}
\author{Mateusz Podmokły}
\date{14 marzec 2024}

\begin{document}
    \maketitle

    \section{Treść zadania}
    \textbf{Zadanie 1.} Wyznacz wielomian interpolacyjny dla punktów
    reprezentujących populację Stanów Zjednoczonych na przestrzeni
    lat. Dane do interpolacji:

    \begin{table}[H]
        \centering
        \begin{tabular}{c | c}
            Rok & Populacja \\
            \hline
            1900 & 76 212 168 \\
            1910 & 92 228 496 \\
            1920 & 106 021 537 \\
            1930 & 123 202 624 \\
            1940 & 132 164 569 \\
            1950 & 151 325 798 \\
            1960 & 179 323 175 \\
            1970 & 203 302 031 \\
            1980 & 226 542 199 \\
        \end{tabular}
    \end{table}
    \subsection*{}
    Rozważ następujące funkcje bazowe $\phi_j(t)$ dla wielomianu,
    gdzie $j=1,..,9$:
    \[
        \phi_j(t)=t^{j-1}
    \]
    \[
        \phi_j(t)=(t-1900)^{j-1}
    \]
    \[
        \phi_j(t)=(t-1940)^{j-1}
    \]
    \[
        \phi_j(t)=\left(\frac{t-1940}{40}\right)^{j-1}
    \]
    Dla najlepiej uwarunkowanej bazy wielomianów wyznacz wielomian
    interpolacyjny na trzy sposoby. Pierwszy polega na rozwiązaniu
    układu równań powstałego z macierzy Vandermonde'a i funkcji bazowych
    \[
        \begin{bmatrix}
            \phi_1(x_1) & \phi_2(x_1) & \phi_3(x_1) & \cdots &
            \phi_n(x_1) \\
            \phi_1(x_2) & \phi_2(x_2) & \phi_3(x_2) & \cdots &
            \phi_n(x_2) \\
            \vdots & \vdots & \vdots & \ddots & \vdots \\
            \phi_1(x_n) & \phi_2(x_n) & \phi_3(x_n) & \cdots &
            \phi_n(x_n)
        \end{bmatrix}
        \begin{bmatrix}
            a_1 \\
            a_2 \\
            \vdots \\
            a_n
        \end{bmatrix}
        =
        \begin{bmatrix}
            y_1 \\
            y_2 \\
            \vdots \\
            y_n
        \end{bmatrix}
    \]

    \subsection*{}
    Następnie oblicz wielomian interpolacyjny Lagrange'a oraz wielomian
    interpolacyjny Newtona i dokonaj ekstrapolacji wielomianu do
    roku 1990. Porównaj otrzymaną wartość ekstrapolacji z prawdziwą
    wartością populacji w roku 1990 wynoszącą 248 709 873 \\
    Na koniec zaokrąglij dane wejściowe do pełnych milionów, ponownie
    oblicz współczynniki wielomianu i porównaj wyniki interpolacji
    z poprzednimi wynikami.

    \section{Specyfikacja użytego środowiska}
    Specyfikacja:

    \begin{itemize}
        \item Środowisko: Visual Studio Code,
        \item Język programowania: Python,
        \item System operacyjny: Microsoft Windows 11,
        \item Architektura systemu: x64.
    \end{itemize}

    \section{Rozwiązanie problemu}
    
\end{document}
